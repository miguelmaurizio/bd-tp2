\section{Parte 1}
Para el dise\~no de la base de documentos solicitada se tomar\'a como base la siguiente porci\'on del modelo realizado para el Trabajo Pr\'actico nro. 1\footnote{Cabe aclarar que se han quitado las FK a entidades que no aparecen en la porci\'on que estamos utilizando, para darle mayor claridad a la lectura}:

\begin{description}
\item[Usuario](\underline{idUsuario}, calle, numero, localidad, telefono, email, tipo)\\
PK=\{idUsuario\}

 \item[SuscripcionRubiOriente](\underline{idSuscripcion}, periodo, \dashuline{idUsuario})\\
PK=\{idSuscripcion\}\\
FK=\{idUsuario\}

 \item[Factura](\underline{idFactura}, periodo, monto, \dashuline{idUsuario})\\
PK=\{idFactura\}\\
FK=\{idUsuario\}

\item[Publicacion](\underline{idPublicacion}, titulo, fecha, precio, \dashuline{tipoPublicacion}, tipoVigencia, tipoVenta, \dashuline{idUsuario})\\
PK=\{idPublicacion\}\\
FK=\{tipoPublicacion, idUsuario\}

 \item[PublicacionFinalizada](\underline{\dashuline{idPublicacion}})\\
PK=\{idPublicacion\}\\
FK=\{tipoPublicacion\}

 \item[TipoPublicacion](\underline{nombre}, comision, costo, nivel, caducidad)\\
PK=\{nombre\}

 \item[Item](\underline{idItem}, \dashuline{idPublicacion}, nombre, tipo)\\
PK=\{idItem\}\\
FK=\{idPublicacion\}

 \item[Producto](\underline{\dashuline{idItem}})\\
PK=\{idItem\}\\
FK=\{idItem\}

 \item[Servicio](\underline{\dashuline{idItem}}, precioXHora)\\
PK=\{idItem\}\\
FK=\{idItem\}

 \item[Compra](\underline{idCompra}, fecha, \dashuline{idUsuario}, \dashuline{idPublicacion}, \dashuline{idCalificacion})\\
PK=\{idCompra\}\\
FK=\{idUsuario, idPublicacion, idCalificacion\}

 \item[Calificacion](\underline{idCalificacion}, valoracionComprador, valoracionVendedor, comentarioComprador, comentarioVendedor)\\
PK=CK=\{idCalificacion\}

\end{description}

que se corresponde con la secci\'on del diagrama que vemos a continuaci\'on

\begin{landscape}
    \begin{figure}[c]
        \frame{\includegraphics[scale=0.40]{imagenes/der-tp2.png}}
        \captionof{figure}{Diagrama Entidad Relaci\'on}
    \end{figure}
\end{landscape}

Desnormalizando esta porci\'on del modelo, generamos los dise\~nos de la base de documentos solicitada:

\begin{itemize}
\item Los documentos correspondientes a las facturas enviadas tendr\'an el siguiente dise\~no:
\begin{verbatim}
{"id","usuario","suscripcionRubi","monto","fecha"}
\end{verbatim}
Los datos para llenarlos se pueden obtener ejecutando la siguiente consulta en la base de SQL:
\begin{verbatim}
SELECT f.idFactura AS id, f.idUsuario AS usuario, 1 AS suscripcionRubi, f.monto, 
f.periodo AS fecha
FROM factura f 
INNER JOIN suscripcionRubiOriente sr ON f.idUsuario = sr.idUsuario
WHERE f.periodo = sr.periodo 
AND f.periodo < date('now','-1 month')
UNION 
SELECT f.idFactura AS id, f.idUsuario AS usuario, 0 AS suscripcionRubi, f.monto, 
f.periodo AS fecha
FROM factura f
WHERE f.idUsuario NOT IN 
(SELECT idUsuario FROM suscripcionRubiOriente WHERE periodo = f.periodo)
AND f.periodo < date('now','-1 month')
\end{verbatim}

\item Los documentos correspondientes a las ventas realizadas tendr\'an el siguiente dise\~no:
\begin{verbatim}
{"id","fecha","idComprador","idVendedor","calificacionComprador",
  "calificacionVendedor","comision","tipoPublicacion"}
\end{verbatim}
Los datos para llenarlos se pueden obtener ejecutando la siguiente consulta en la base de SQL:
\begin{verbatim}
SELECT c.idCompra AS id, c.fecha AS fecha, c.idUsuario AS idComprador,
p.idUsuario AS idVendedor, cal.valoracionComprador AS calificacionComprador, 
cal.valoracionVendedor AS calificacionVendedor, (tp.comision * p.precio) AS comision, 
"servicio" AS tipoDePublicacion
FROM compra c, calificacion cal, publicacion p, publicacionFinalizada pf, 
tipoPublicacion tp, item it
WHERE c.idPublicacionFinalizada = pf.idPublicacion AND pf.idPublicacion = p.idPublicacion 
AND p.nombreTipoPublicacion = tp.nombre 
AND it.idPublicacion = c.idPublicacionFinalizada AND it.tipo = 1
AND c.idCalificacion = cal.idCalificacion
AND c.fecha < date('now','-1 month')
UNION
SELECT c.idCompra AS id, c.fecha AS fecha, c.idUsuario AS idComprador, 
p.idUsuario AS idVendedor, cal.valoracionComprador AS calificacionComprador, 
cal.valoracionVendedor AS calificacionVendedor, 
(tp.comision * p.precio) AS comision, "producto" AS tipoDePublicacion
FROM compra c, calificacion cal, publicacion p, publicacionFinalizada pf, 
tipoPublicacion tp, item it
WHERE c.idPublicacionFinalizada = pf.idPublicacion AND pf.idPublicacion = p.idPublicacion 
AND p.nombreTipoPublicacion = tp.nombre 
AND it.idPublicacion = c.idPublicacionFinalizada AND it.tipo = 0
AND c.idCalificacion = cal.idCalificacion
AND c.fecha < date('now','-1 month')
UNION
SELECT c.idCompra AS id, c.fecha AS fecha, c.idUsuario AS idComprador, 
p.idUsuario AS idVendedor, cal.valoracionComprador AS calificacionComprador, 
cal.valoracionVendedor AS calificacionVendedor, 
(tp.comision * p.precio) AS comision, "mixto" AS tipoDePublicacion
FROM compra c, calificacion cal, publicacion p, publicacionFinalizada pf, 
tipoPublicacion tp, item it
WHERE c.idPublicacionFinalizada = pf.idPublicacion AND pf.idPublicacion = p.idPublicacion 
AND p.nombreTipoPublicacion = tp.nombre 
AND it.idPublicacion = c.idPublicacionFinalizada AND it.tipo <> 0 AND it.tipo <> 1
AND c.idCalificacion = cal.idCalificacion
AND c.fecha < date('now','-1 month')
\end{verbatim}

\end{itemize}
