\section{Conclusi'on}

En este trabajo estudiamos los modelos de bases de datos no relacionales. Hemos realizado un diseño en el paradigma $document$ $oriented$ pensado para optimizar querys particulares,  donde luego utilizamos la técnica de MapReduce. También utilizamos $sharding$ para que en un escenario hipotético los accesos a los datos sean eficientes. Se puede observar que el modelado de las bases NoSQL es sencillo y permite que las consultas sean mas simples que en las tradicionales SQL ya que el diseño se orienta a responder consultasde nuestro interes.\\

Tanto MapReduce como Sharding ponen en evidencia la capacidad de estas bases para un escenario de cómputo distribuido, en los cuales los datos se encuentran en varias computadoras.\\

En particular observamos que sharding, bajo ciertas hipotesis en la distribucion de datos ingresados, puede balancear la carga de datos en distintos nodos. Esto supone una mejor utilización del hardware e incluso un mejor tiempo de respuesta para el usuario.
