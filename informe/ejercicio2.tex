\section{Ejercicio 2}

En la sección previo desnormalizamos la base de datos SQL. En esta parte vamos a realizar diversas consultas utilizando $Map$ $Reduce$.

\subsection{Importe total de ventas por usuario}

Por cada factura registrada, vemos el usuario y el importe, luego calculamos el total paralelizando por usuario: \\

\begin{verbatim}
var map1 = function(){
    emit(this["id"],this["monto"])
}

var reduce1 = function(key,values){
    return Array.sum(values)
}
\end{verbatim}

\subsection{reputación histórica de cada usuario}
Por cada venta devolvemos tanto el puntaje del comprador como del vendedor. Luego juntamos todas las calificaciones en un mismo nodo.


Entendemos que la $reputacion$ $historica$ se refiere al promedio de las calificaciones. Si fuera otra la función, no sería un problema dado que tenemos el historial completo de cada usuario en cada reduce y podríamos implementar otra diferente.

\begin{verbatim}
var map2 = function(){
    emit(this["idComprador"],this["calificacionComprador"])
    emit(this["idVendedor"],this["calificacionVendedor"])
}

var reduce2 = function(key,values){
    return Array.sum(values) / values.length
}
\end{verbatim}

\subsection{Operaciones con comisión más alta}

En este caso debemos saber cuál es el valor de la comisión más alta, la forma que encontramos para resolver es enviar todas las operaciones (en realidad solo el id y la comisión) a un solo nodo y ahi calcular las oeraciones con comisión máxima.

\begin{verbatim}
var map3 = function(){
    var pair = {"comision": this["comision"], "id": this["id"]};
    emit(0, pair)
}

var reduce3 = function(key,values){
    var max = 0;
    for(var i =0; i < values.length; i++){
        max = Math.max(max, values[i]["comision"]);
    }
    res = [];
    for(var i =0; i < values.length; i++){
        if(values[i]["comision"] === max) {
            res.push(values[i]["id"])
        }
    }	
    return {"res": res};
}
\end{verbatim}

\subsection{Monto total facturado por año}

Revisamos las facturas, separamos por año y luego sumamos cada monto. Notemos que el reduce es solamente sumar, que es lo mismo que usamos en el reduce de la consulta 1.

\begin{verbatim}
var map4 = function(){
    var s = new Date(
                this["fecha"].replace( /(\d{2})\/(\d{2})\/(\d{4})/, "$2/$1/$3")
            ).getFullYear();
    emit(s, this["monto"]);
}

var reduce4 = reduce1
\end{verbatim}

\subsection{Monto total facturado por año con suscripción Rubí}

Esta consulta es idéntica a la consulta 4, con la diferencia que antes de emitir en el map debemos chequear que en la factura tenga suscripción a Rubí del oriente.

\begin{verbatim}
var map5 = function(){
    if(this["suscripcion"] === "Rubí"){
        var s = new Date(
                this["fecha"].replace( /(\d{2})\/(\d{2})\/(\d{4})/, "$2/$1/$3")
            ).getFullYear();
        emit(s, this["monto"])		
    }
}

var reduce5 = reduce1

\end{verbatim}

\subsection{Total publicaciones por tipo}

Revisamos las ventas y emitimos el tipo, luego contamos cuantas aparecen de cada una.

\begin{verbatim}
var map6 = function(){
    emit(this["tipoPublicacion"], 1)		
}

var reduce6 = reduce1

\end{verbatim}
