\section{Introducci'on}

Las bases de datos relacionales SQL cuentan con muchas ventajas pero su utilización puede ser imposible en el caso de tener grandes volúmenes de datos. En cambio, las bases de datos NoSQL tienen la principal ventaja de manejar mucha más inforcación debido a la posibilidad de escalar en forma horizontal, lo cual no es posible con las bases de datos tradicionales. Es decir, para aumentar la cantidad de datos que procesa una base de datos relacional debemos tener una computadora mejor. 

Por otro lado, al estar utilizando una base NoSQL se puede incrementar la capacidad mediante una mayor cantidad de computadoras. Esto implica que en contextos donde la cantidad de datos aumenta rápidamente debemos utilizar una arquitectura distribuida, forzándonos a usar una base de datos NoSQL. 

En este trabajo estudiamos la bases de datos NoSQL. Particularmente aquellas de la familia orientada a documentos, utilizando la tecnología $MongoDB$.

En la primera parte, utilizando el problema del trabajo práctico 1, estudiamos y analizamos el diseño de los documentos a utilizar con el propósito de responder queries específicas. No solo creamos la base SQL, sino que también implementamos la migración de SQL a NoSQL.


En la segunda parte, utilizamos el esquema $Map$ $Reduce$ para realizar consultas sobre los datos, permitiendo parelelismo en su ejecución.


Por último, exploramos e investigamos la técnica de \textit{sharding}.
